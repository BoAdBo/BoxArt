\documentclass{beamer}
\usepackage{verbatim}
\usepackage{fancyvrb}
\usepackage{xeCJK}
\usepackage{enumerate}
\usepackage{mathtools}
\usepackage{hyperref}
\setmainfont{DejaVu Sans}
\usepackage[T1]{fontenc}

\usetheme{Warsaw}

%\usetheme[progressbar=frametitle]{metropolis}
\setbeamertemplate{frame numbering}[fraction]
%\useoutertheme{metropolis}
%\useinnertheme{metropolis}
%\usefonttheme{metropolis}
\usecolortheme{spruce}
\setbeamercolor{background canvas}{bg=white}


\definecolor{mygreen}{rgb}{.125,.5,.25}
\usecolortheme{crane}
%\usecolortheme[named=mygreen]{structure}

\title{艇员管理系统}
\subtitle{实验汇报}
\author{杨麒平 15336220}
\institute{rowing crew}
\date{\today}

\begin{document}
%\metroset{block = fill}

\begin{frame}
  \titlepage
\end{frame}

\begin{frame}{动机}
  \only<1> {
    \begin{block}{关于这个系统}
    艇员管理系统是一个能够发布训练计划,统计训练记录,并能通过统计得到的训练记
    录进行反馈,为队员提供参考的系统。
  \end{block}


    考虑下面一个训练计划:

  \begin{enumerate}
  \item {深蹲跳 1min测试 3组}
  \item {卧拉 25kg 1min测试 3组 女生 15kg}
  \item {卧推 25kg 1min测试 3组 女生 15kg}
  \item {仰卧两头起 1min测试 3组}
  \item {200米冲刺跑 4组}
  \item {慢跑五圈放松}
  \end{enumerate}}

  \only<2> {
    example from wechat

    这样发布的计划和记录会在聊天记录里面分布得比较散,因此对于统计训练计划以及记
    录并进行反馈的这项工作会比较繁琐,而且数据也难免丢失,队员训练的训练反馈效果
    不会很好。
  }

  \only<3> {
    \begin{block}{解决方案}
    这个时候数据库应用能够简化这个工作,并给出更好的反馈。
    \end{block}
  }

  % 一个体育团队通过收集每天的训练情况来定期给队员提供积
  %   极科学的反馈,并以此能够修正训练计划,从而使训练更加高效有序。可是现实团队里面在
  %   即时通讯软件(wechat)里面发布和收集训练记录,或者说用纸来记录数据。这样的做法费
  %   时费力而且数据也不完整,即使通信软件里面有相当强大的搜索功能。但是如果每次进行数
  %   据收集都是对聊天记录进行搜索的话,数据难免丢失(数据格式不一致等原因)因此开发一
  %   个数据库应用来进行队内事务管理是相当有必要的。

  % example

  
  % One major point: State it, develop it, and repeat it.

  % Argument, evidence, and a conclusion.

  % Organize information in a highly structured fashion.

  % Clear topic sentences.
  
  % Neural machine translation is a newly emerging approach to machine translation.

  % Compress all the necessary information of a source sentence into a
  % fixed-length vector. This may make it difficult for the neural network to cope
  % with long sentences, especially those that are longer than the sentences in
  % the training corpus.

  
\end{frame}

\begin{frame}{数据库系统主要设计}
  用户信息,成员信息,训练计划信息,训练信息,艇支信息,支出信息。

  其中用户系统需要注意的问题是密码的储存要hash,不能用明文存储。

  成员系统,具体难度不大。艇支管理和支出管理不是实现的主要目的。

  重点则是训练计划以及训练记录信息。
  
\end{frame}

\begin{frame}{设计难点}

  \only<1> {训练记录和训练计划的设计
    仔细看训练计划,主要有三个部分,时间,训练项目和指标。

    一个计划可以有多个项目
    比如:深蹲,卧拉,卧推等。
   
    其中训练项目有着各自不一样的指标。
    比如:

    深蹲的指标是重量,次数,组数等。
    跑步的指标是距离,时间等。

    所以指标其实是一个多值属性,包括\{指标名称,指标数目,指标要求\}
  }
  
  \only<2>{
    不够好的设计

    不好的原因,不过动态,不够灵活。
  }
  
  \only<3>{
    解决方案
  }
    
\end{frame}
  
\begin{frame}{展示}

\end{frame}

\end{document}
%%% Local Variables:
%%% TeX-engine: xetex
%%% end: