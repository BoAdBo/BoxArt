\documentclass{beamer}
\usepackage{verbatim}
\usepackage{fancyvrb}
\usepackage{xeCJK}
\usepackage{enumerate}
\usepackage{mathtools}
\usepackage{hyperref}
\setmainfont{DejaVu Sans}
\usepackage[T1]{fontenc}

\usetheme{Warsaw}

%\usetheme[progressbar=frametitle]{metropolis}
\setbeamertemplate{frame numbering}[fraction]
%\useoutertheme{metropolis}
%\useinnertheme{metropolis}
%\usefonttheme{metropolis}
\usecolortheme{spruce}
\setbeamercolor{background canvas}{bg=white}


\definecolor{mygreen}{rgb}{.125,.5.25}
\usecolortheme{crane}
\usecolortheme[named=mygreen]{structure}

\title{crew-management-system}
\subtitle{Subtile Here}
\author{杨麒平}
\institute{}
\date{\today}

\begin{document}
%\metroset{block = fill}

\begin{frame}
  \titlepage
\end{frame}

\begin{frame}
  One major point: State it, develop it, and repeat it.

  Argument, evidence, and a conclusion.

  Organize information in a highly structured fashion.

  Clear topic sentences.
  
  Neural machine translation is a newly emerging approach to machine translation.

  Compress all the necessary information of a source sentence into a
  fixed-length vector. This may make it difficult for the neural network to cope
  with long sentences, especially those that are longer than the sentences in
  the training corpus.

  
\end{frame}


\end{document}
