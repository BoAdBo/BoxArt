\chapter{Crew Management System}

\section{开发环境与开发工具}

\section{系统需求分析}% need further investigation, query somebody maybe


A crew management system can store the training record and provide a great feed
back for the training crew. Our crew is training on a daily basis, one finds
that by comparing and contrast, would make a positive and huge impact on the
confidence and momentum of the training individuals as well as crew members.
Therefore developing a crew management system would greatly enhance the efficiency
in terms of designing training plans and logging.

\section{功能需求分析}
%making the blow text smaller perhaps?
\textbf{A basic crew management system would require the following N functionalities:}
\begin{enumerate}
\item
  {user management
    \begin{enumerate}
    \item{user adding\\
        include 3 kinds of member type: Crew Coach, Crew Leader, Crew member\\
        
        Crew Coach can announce the plans for this year: have access to
        modifying the calender: dates for group extra training program, dates for out
        field races, etc..\\
        
        Crew Leader can design and publish training plans(in the form of
        notice-board) and specify the training program for the day. Have access
        to team budgets, logging of team budgets.\\
        
        Crew members can update their status on the training program of the day.
        How many push-ups and squats(training programs of the day), how much time
        for the 10 laps running, etc..\\

        Crew member information: Entrance date, height, weight, age.
        
        To be continued...
        }
      \item{user deletion\\
          the deletion of the mentioned crew member types. Cascade deletion(deleting
          team members would also delete their training records)
        }
      \item{password modification
          add password, update password. Involves registration.\\
        }
    \end{enumerate}
  }
\item {
    training log management
    \begin{enumerate}
      \item{}
    \end{enumerate}
  }
  
\item{
    training log queries
    \begin{enumerate}
      \item{}
    \end{enumerate}
  }
  
\end{enumerate}
\section{系统设计}

\section{系统功能的实现}

\section{课程设计心得体会}
% 此处输入本文档的面向读者对象

% \section{参考文档}
% 此处输入在撰写本文档过程中所使用到的资源列表。

% \begin{itemize}
%     \item 《xxx系统技术协议书》,2016年12月
%     \item 《xxx系统需求说明文档》,2017年6月
%     \item 《xxx系统详细设计文档》,2017年6月
% \end{itemize}


% \section{术语与缩写解释}

% 本文档中使用到的术语缩写及其解释如下表\ref{shuyu}所示。

% \begin{table}[htb]
% \centering
% \label{shuyu}
% \begin{tabular}{p{2cm}|p{8cm}}
% \hline\hline

% \textbf{术语缩写} & \textbf{解释} \\
% \hline\hline

% 术语缩写 & 解释 \\
% \hline

% 术语缩写 & 解释 \\
% \hline

% 术语缩写 & 解释 \\
% \hline

% \hline\hline
% \end{tabular}
% \caption{本文档中出现的术语缩写及解释}
% \end{table}
